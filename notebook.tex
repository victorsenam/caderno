\documentclass[a4paper,oneside]{article}
\usepackage[utf8]{inputenc}
\usepackage[english]{babel}
\usepackage{listings}
\usepackage{color}
\usepackage{verbatim}
\usepackage{amssymb}
\usepackage[inline]{enumitem}
\usepackage{pxfonts}

\usepackage[a4paper,top=2cm,bottom=1cm,left=1cm,right=1cm]{geometry}
\usepackage{fancyhdr}
\pagestyle{fancy}
\renewcommand{\sectionmark}[1]{\markright{#1}}
\fancyhf{}
\rhead{\fancyplain{}{\nouppercase{\rightmark}}, page \bfseries\thepage}
\lhead{University of São Paulo}

\usepackage{titlesec}
\titlespacing*{\section}
{0pt}{2ex}{1ex}

\definecolor{dkgray}{rgb}{0.4,0.4,0.4}
\definecolor{gray}{rgb}{0.6,0.6,0.6}

\lstset{frame=tb,
  language=c++,
  aboveskip=.1em,
  belowskip=.1em,
  showstringspaces=false,
  columns=flexible,
  basicstyle={\small\ttfamily},
  numbers=none,
  keywordstyle=\bfseries,
  commentstyle=\color{gray},
  morekeywords={num, cood},
  breaklines=true,
  breakatwhitespace=true,
  tabsize=4,
  showtabs=true,
  tab={\color{dkgray}{\tiny $\vartriangleright$}\hfill},
  moredelim=**[is][\color{red}]{@}{@}
}

\title{ACM ICPC Reference}
\author{University of São Paulo}

\begin{document}
\maketitle
\thispagestyle{fancy}
\tableofcontents
\newpage

\begin{enumerate*}
 \item workspaces/teclado 
 \item .vimrc and .bashrc
 \item temp.cpp
\end{enumerate*}

\begin{lstlisting}
syntax on
colo evening
set ai si noet ts=4 sw=4 sta sm nu rnu so=7 t_Co=8
imap {<CR> {<CR>}<Esc>O
\end{lstlisting}

\lstinputlisting{hashify.py}
\section{Geometry}
\subsection{Base}
\lstinputlisting{src/geometry/basic.cpp}

\subsection{Advanced}
\lstinputlisting{src/geometry/algorithms.cpp}

\subsection{3D}
\lstinputlisting{src/geometry/3d.cpp}

\section{Graphs}
\subsection{Dinic}
\lstinputlisting{src/dinic.cpp}

\subsection{MinCost MaxFlow}
\lstinputlisting{src/min_cost.cpp}

\subsection{Cycle Cancelling}
\lstinputlisting{src/cycle_cancel.cpp}

\subsection{Hungarian}
\lstinputlisting{src/hungarian.cpp}

\section{Structures}
\subsection{Ordered Set}
\lstinputlisting{src/orderedset.cpp}

\subsection{Treap}
\lstinputlisting{src/treap.cpp}

\subsection{Envelope}
\lstinputlisting{src/envelope.cpp}

\subsection{Centroid}
\lstinputlisting{src/centroid.cpp}

\subsection{Link Cut Tree}
\lstinputlisting{src/linkcut.cpp}

\subsection{Splay Tree}
\lstinputlisting{src/splay.cpp}

\section{Strings}
\subsection{Suffix Tree}
\lstinputlisting{src/sufftree.cpp}

\subsection{Z-function}
\lstinputlisting{src/zfunction.cpp}

\subsection{Manacher}
\lstinputlisting{src/manacher.cpp}

\section{Math}
\subsection{FFT}
\lstinputlisting{src/math/fft.cpp}

\subsection{Discrete FFT}
\lstinputlisting{src/math/fft_finite.cpp}

\subsection{Linear System Solver}
\lstinputlisting{src/linsys.cpp}

\subsection{Simplex}
\lstinputlisting{src/simplex.cpp}

\subsection{Zeta}
\lstinputlisting{src/zeta.cpp}

\subsection{Zeta Disjoint Or}
\lstinputlisting{src/zeta_dor.cpp}

\subsection{Miller-Rabin}
\lstinputlisting{src/millerrabin.cpp}

\subsection{Pollard-Rho}
\lstinputlisting{src/pollard_rho.cpp}

\section{Old Solutions}
\subsection{Ceiling Function}
\lstinputlisting{src/easy/ceiling.cpp}

\subsection{Secret Chamber at Mount Rushmore}
\lstinputlisting{src/easy/secret.cpp}

\subsection{Need for Speed}
\lstinputlisting{src/easy/speed.cpp}

\subsection{Amalgamated Artichokes}
\lstinputlisting{src/easy/amalgated.cpp}

\subsection{Low Power}
\lstinputlisting{src/easy/low_power.cpp}

\section{Anotações}
\subsection{Intersecção de Matróides}
Sejam~$M_1 = (E, I_1)$ e~$M_2 = (E, I_2)$ matróides. Então
$ \max\limits_{S \in I_1 \cap I_2}{|S|} = \min\limits_{U \subseteq E}{r_1(U) + r_2(E \setminus U)}. $

\subsection{Möebius}
Se~$F(n) = \sum\limits_{d | n}{f(d)}$, então
$f(n) = \sum\limits_{d | n}{\mu(d) F(n / d)}.$

\subsection{Burnside}
Seja~$A \colon GX \rightarrow X$ uma ação. Defina:
\begin{itemize}
\item $w \coloneqq $ número de órbitas em~$X$.
\item $S_x \coloneqq \{g \in G \mid g \cdot x = x \}$
\item $F_g \coloneqq \{x \in X \mid g \cdot x = x \} $
\end{itemize}

Então $ w = \frac{1}{|G|} \sum\limits_{x \in X}{|S_x|} = \frac{1}{|G|} \sum\limits_{g \in G}{|F_g|}. $

\subsection{Landau}
Existe um torneio com graus de saída~$d_1 \leq d_2 \leq \ldots \leq d_n$ sse:
\begin{itemize}
\item $d_1 + d_2 + \ldots + d_n = {n \choose 2}$
\item $d_1 + d_2 + \ldots + d_k \geq {k \choose 2} \quad \forall 1 \leq k \leq n.$
\end{itemize}
Para construir, fazemos~1 apontar para~$2, 3, \ldots, d_1 + 1$ e seguimos recursivamente.

\subsection{Erdös-Gallai}
Existe um grafo simples com graus~$d_1 \geq d_2 \geq \ldots \geq d_n$ sse:
\begin{itemize}
\item $d_1 + d_2 + \ldots + d_n$ é par
\item $\sum\limits_{i = 1}^k{d_i} \leq k(k-1) + \sum\limits_{i=k+1}^n{\min(d_i, k)} \quad \forall 1 \leq k \leq n$.
\end{itemize}
Para construir, ligamos~1 com~$2, 3, \ldots, d_1 + 1$ e seguimos recursivamente.

\subsection{Gambler's Ruin}
Em um jogo no qual ganhamos cada aposta com probabilidade~$p$ e perdemos com probabilidade~$q \coloneqq 1 - p$, paramos quando ganhamos~$B$ ou perdemos~$A$. Então~$\mathit{Prob}(\textnormal{ganhar B}) = \frac{1 - (p/q)^B}{1 - (p/q)^{A+B}}$.

\subsection{Extra}
\newcommand{\Fib}{\mathit{Fib}}
\begin{itemize}
\item $\Fib(x + y) = \Fib(x + 1) \Fib(y) + \Fib(x) \Fib(y - 1)$
\end{itemize}

\end{document}

